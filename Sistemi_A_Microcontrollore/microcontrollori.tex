\documentclass[12pt, a4paper]{article}

\title{Microcontrollori}
\date{}


	\usepackage[margin=1in]{geometry}  % set the margins to 1in on all sides
	\usepackage{graphicx}              % to include figures
	\usepackage{amsmath}               % great math stuff
	\usepackage{amsfonts}              % for blackboard bold, etc
	\usepackage{amsthm}                % better theorem environments
	\usepackage{wrapfig}			   % for images formatting
	\usepackage{booktabs}
	\usepackage{caption}			   % for labeling images
	\usepackage{bigints}			   % for bigger integral symbols	
	\usepackage{subcaption}			   % subfigures functions
	
	% various theorems, numbered by section
	
	\newtheorem{thm}{Theorem}[section]
	\newtheorem{lem}[thm]{Lemma}
	\newtheorem{prop}[thm]{Proposition}
	\newtheorem{cor}[thm]{Corollary}
	\newtheorem{conj}[thm]{Conjecture}
	
	\DeclareMathOperator{\id}{id}
	
	\newcommand{\bd}[1]{\mathbf{#1}}  % for bolding symbols
	\newcommand{\RR}{\mathbb{R}}      % for Real numbers
	\newcommand{\ZZ}{\mathbb{Z}}      % for Integers
	\newcommand{\col}[1]{\left[\begin{matrix} #1 \end{matrix} \right]}
	\newcommand{\comb}[2]{\binom{#1^2 + #2^2}{#1+#2}}
	
\begin{document}
	\maketitle
	
	\section*{Lezione 2}
	\subsection*{Leaf procedure(?)}
	ISA: l'insieme di regole che gestiscono l'nterazione tra Hardware e Software
	``Smaller is Faster''
	Abbiamo 2 formati, il formato 'R' ed il formato 'L'\\
	(Figure 1) \\
	
	Formato R(32 bit): Si usano 5 bit perché servono ad indirizzare 32 diversi registri.\\
		( OPCODE(6 bit)-rs-rt-rd-sh(shift)-FUNCT)\\ \\
	Formato I: OPCODE-rs-rd-const\\ \\
	MIPS Code: \\ 
	leaf\_example:\\
	1. addi \$sp, \$sp, -4  \\ 
	2. sw \$s0, 0(\$sp)  \quad(Save \$s0 on stack) \\
	-------------------------------------------------- \\
	3. add \$t0, \$a0, \$a1 \\
	4. add \$t1, \$a2, \$a3 \\
	5. sub \$s0, \$t0, \$t1     \quad(Procedure body) \\
	-------------------------------------------------- \\
	6. add \$v0, \$s0, \$zero    \quad(Result) \\
	-------------------------------------------------- \\
	7. lw \$s0, 0(\$sp) \\
	8. addi \$sp, \$sp, 4 	\quad(Restore \$s0) \\
	-------------------------------------------------- \\
	9. jr \$ra   \quad(Return) \\ \\
	
	Ogni microprocessore ha un Program Counter che viene aumentato di 4 ad ogni ciclo. Nel caso ci sia un jump. Nel caso di un jmp, il PC sale alla prima istruzione di una possibile funzione. La prima istruzione (in asm) della prima istruzione sarà sicuramente un' ''addi'' poiché il PC viene `forzato' ad assumere il valore della posizione di quella funzione. \\
	Local data allocated by callee \\
	Procedure frame (?) \\ \\
	
	\subsection*{Memory Layout}
	\begin{itemize}
		\item Text: program code
		\item Static data: global variable (static var in C, const arrays and strings) \\
			\$gp initialized to address allowing $\pm$offsets into this segments
		\item Dynamic data: heap (malloc in C, new in Java)
		\item Stack: automatic storage
	\end{itemize} 
	
	\subsection*{Jump Addressing}
	Jump (j and jal) targets could be anywhere in text segment. \\

	Encode full address in instruction \\
	OP-address (6 bits-26 bits) (Fig.3)\\
	\textbf{Slide: Addressing Mode Summary} \\\\
	
	\section*{Chap. 4  The processor}
	
	Exe (execution time) = CCT (clock cicle time) $\cdot$ IC $\cdot$ CPI(= 1) \\
	Control is combinatory
	Sequenziale è qualcosa che ha uno stato e ne deve tener traccia \\
	Combinatorio è qualcosa il cui output dipende unicamente dagli input \\ 
	RegWrite: Quando è alto permette di scrivere, quando è basso non scrive \\

	Nelle connessioni c'è sempre voltaggio, non esiste un modo per disconnettere il collegamente, ecco perché serve un controllore per verificare che un certo collegamento sia in uso.	\\
	
	Due numeri sono uguali se la loro sottrazione vale 0, motivo per cui beq(branch if equal) si avvale dell'uso della ALU per l'esecuzione dell'istruzione.
\\  L'istruzione(32 bit) nel datapath viene spacchettata ed inviata. \\
	Per la ALU si fa un controllo ``gerarchico'', il valore si presenta come una macroclasse di operazioni, ed i 6 bit meno significativi vanno nel controllo della ALU.\\
	Il metodo della pipeline permette di ridurre CCT e CPI contemporaneamente. Mentre una fa l'instruction fetch, l'altra fa l'accesso ai registri, mentre una accede ai registri l'altra utilizza l'ALU. E' una forma molto low-level di parallelismo.
	

\end{document}